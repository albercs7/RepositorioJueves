\documentclass[10pt,a4paper]{book}
\usepackage[utf8]{inputenc}
\usepackage[spanish]{babel}
\usepackage{amsmath}
\usepackage{amsfonts}
\usepackage{amssymb}
\usepackage{makeidx}
\usepackage{graphicx}
\usepackage{lmodern}
\usepackage{kpfonts}
\usepackage{wrapfig}
\usepackage[left=2cm,right=2cm,top=2cm,bottom=2cm]{geometry}
\author{{\LARGE Alberto Castellon Sanchez}}
\title{{\Huge Este ejercicio es un pepino}}
\begin{document}
\maketitle

\tableofcontents
\part{LA LISTA}
\chapter*{Lista}
{\LARGE \textbf{Atentos que viene una lista to guapa.}}
\\
\begin{enumerate}
\item El uno es un número muy suyo, siempre va primero.\\
\item El dos va después.\\
\item Tres son multitud, aunque no siempre.\\
\item Cuatro ya son demasiados.\\
\item Este lleva premio.\\
\end{enumerate}
{\LARGE \textbf{Ahora otra pero con puntos.}}\\
\begin{itemize}
\item Punto  
\item Otro punto
\item Otro punto
\item Opro tunto
\item Ptro ounto
\item Ptrn uontt
\end{itemize}

\newpage
\part{LA IMAGEN}
\chapter*{Imagenes}
{\LARGE \textbf{A continuación unas imagenes de mis idolos.}}\\\\
\begin{center}
\begin{wrapfigure}{r}{0.5\textwidth} 
    \includegraphics[width=0.4\textwidth]{Escritorio/MichaelKnightOS.jpg}
    \caption{Mikel Knight}
    El personaje protagonista de la serie, Michael Knight (interpretado por el actor David Hasselhoff), es un defensor de los pobres y desamparados que combate la injusticia conduciendo un prototipo de automóvil de alta tecnología. El automóvil, llamado KITT (Knight Industries Two Thousand), incorpora una computadora central y no sólo es autoconsciente y altamente inteligente sino que hasta tiene la capacidad de hablar como una persona normal.
    \label{fig:databaseUserTable}
\end{wrapfigure}
\includegraphics[scale=0.2]{Escritorio/maxresdefault.jpg}\\ El personaje del título es Gordon Shumway, un extraterrestre apodado A.L.F “Alien Life Form”, del inglés «Forma de vida Extraterrestre») y “Amorfismo Lejano Fantástico” al doblaje de español latino. Nació el 28 de octubre de 1756, en la región Lower East del planeta Melmac, que a su vez estaba localizado 6 pársecs más allá del supercúmulo Hydra-Centaurus, tenía cielo verde, pasto azul y agua naranja.
\end{center}
\part{LAS FORMULAS}
\chapter*{Formulas} 
{\LARGE \textbf{Para terminar unas formulillas matemáticas}}\\\\
$$ \sum_{k=1}^n=\sqrt{\frac{E^{n}+8}{B_{m}}}+\sqrt[5]{mi\ prima}\ \ (Ley\ de\ agromenawer)$$
$$ \prod_{k=1}^n=\sqrt{\frac{uno\ que\ va}{otro\ que\ llega}-\overrightarrow{L.Casei\ imunitas}}\ \ (Ley\ actimel) $$
\begin{tabular}{|c|c|c|c|c|c|}
\hline 
1 & 2 & 3 & 4 & 5 & 6 \\ 
\hline 
2 & 3 & 4 & 5 & 6 & 7 \\ 
\hline 
3 & 4 & 5 & 6 & 7 & 8 \\ 
\hline 
4 & 5 & 6 & 7 & 8 & 9 \\ 
\hline 
\end{tabular} 
\end{document}